\documentclass[submission]{eptcs}
\setlength{\parindent}{0pt}
\providecommand{\event}{} % Name of the event you are submitting to

\title{Videojuego educativo para enseñar algoritmos relacionados a grafos}
\author{ Alonso Utreras\\
Profesor Guía: Iván Sipirán
\institute{Departmento de Ciencias de la Computación\\
Universidad de Chile\\
Santiago, Chile}
\email{alonso.utreras@ug.uchile.cl}
\email{autreras@dcc.uchile.cl}
}
\begin{document}
\maketitle

\section{Introducción}
 
Una forma común de enseñar y aprender es acompañar la información con dibujos o gráficos. La presente investigación
busca llevar las instrucciones llevadas a cabo por los algoritmos relacionados a grafos a una
representación visual e interactiva para que los estudiantes de computación puedan aprender acompañándose
con esta herramienta. 

El trabajo consiste en presentarle un videojuego donde se muestren grafos, en que el jugador
debe ejecutar las instrucciones de estos algoritmos, apoyándose en elementos visuales. El objetivo de esta investigación de tesis es 
determinar si se ven diferencias en los niveles de motivación percibidos por el estudiantado y en el entendimiento
de los algoritmos, medido a través de tareas de programación. El público objetivo son estudiantes
que estén en su primer año de ciencias de la computación (CS por sus siglas en inglés).

\section{Trabajo relacionado} 


Un videojuego es una forma de aprendizaje activo, pues el proceso de enseñanza no se desarrolla partiendo por un profesor
exponiendo frente a un estudiante. En este caso, quien aprende debe ejectuar pasos y
participar en alguna actividad a través de la cual se construye el conocimiento. En la reseña realizada por Hartikainen et al. \cite{active_learning_review}
se enumeran justificaciones para el aprendizaje activo:  mejores resultados, recomendaciones políticas y las nuevas demandas de la vida laboral actual,
como capacidades de comunicación o descubrimiento por cuenta propia. 

Los videojuegos serios (Serious Gaming) son una forma de aprendizaje activo. Un trabajo de Bell y Gibson \cite{evaluation_of_games_for_teaching_cs} indican que juegos
educacionales son más efectivos que las clases, lecturas, videos y tareas. Se resume que el uso de juegos resulta en 9\% mejor retención, 11\% mejor conocimiento fáctico,
14\% mejor conocimiento basado en habilidades y 20\% mayor autoeficacia. Sin embargo, recomiendan que los juegos deben ser acompañados
de otras formas de enseñanza, así como hacer actividades post juego donde se le pregunta al estudiantado cómo se relacionan los juegos a la materia.


Bell y Gibson \cite{evaluation_of_games_for_teaching_cs} identificaron y clasificaron videosjuegos de Ciencias de la Computación (CS),
considerando un total de 41 videojuegos. Uno de ellos, Map Coloring, se trata sobre grafos, tomando el tema de coloreo de grafos.
Se realizó una búsqueda del juego a la fecha (2022), pero no se encontró ningún material al respecto.

Kiili y su equipo \cite{using_videogames_maths} analizaron el uso de videojuegos en enseñanza
y evaluación en matemáticas a través de los títulos ``Semideus'' y "Wuzzit Trouble". A través
de estos, llegaron a la conclusión de que es posible utilizar videojuegos para enseñar y
evaluar al mismo tiempo. Además, caracterizaron estadísticamente las diferencias producidas por el uso
de videojuegos entre resultados de un pre test y un post test.

En el trabajo realizado por Zhao y Shute \cite{video_game_foster_computational_thinking} se enumeran ejemplos de videojuegos pensados para enseñar programación,
tales como Wu's Castle \cite{wuscastle}, CodeCombat \cite{CodeCombat}, CodeSpell \cite{codespells}, MiniColon \cite{minicolon},
tales ejemplos utilizan programación con texto. Sin embargo, también hay numerosos ejemplos que utilizan programación por
bloques, como LightBot \cite{LightBot}, Scratch \cite{maloney2010scratch}, \cite{scratch} y RoboBuilder \cite{RoboBuilder}, los
cuales abstraen el trabajo de aprender una sintaxis relacionada a los lenguajes de programación.
Sin embargo, el impacto de estos videojuegos no ha sido evaluado en muchos casos. En los casos documentados, se cuenta con
muestras pequeñas, además de que se trata principalmente de evaluaciones puramente cualitativas \cite{video_game_foster_computational_thinking},
\cite{effectiveness_gbl}.

Kiili y su equipo, Bell et al., Giani y otros \cite{petri2018method} están de acuerdo en que no hay sistematización en la evaluación de
videojuegos educativos. En efecto, hay juegos serios que ni siquiera se denominan o consideran como tales
\cite{evaluation_of_games_for_teaching_cs}. Por otra parte, no existe una forma estándar de evaluarlos, razón por
la cual Petri, Giani y otros \cite{petri2018method} crearon el modelo MEEGA+ (Model for the Evaluation of Educational
Games and EGameFlow scale).

Entre las faltas mencionadas al momento de crear videojuegos, se mencionan las faltas de 1) Definición de un objetivo
de evaluación; 2) Diseño de investigación; 3) Programa de medición; 4) Instrumentos de recolección de datos y 5) Métodos
de análisis de datos. Un ejemplo de falta de sistematización: una práctica común al analizar estas herramientas son los 
comentarios informales por parte del estudiantado \cite{petri2018method}.



\subsection{Propiedades que debería tener el juego}

Según Gibson y Bell \cite{evaluation_of_games_for_teaching_cs}, el juego debería ser divertido, acotado en tiempo y espacio (separado), incierto
(Su desarrollo no está predeterminado), gobernado por reglas y ficticio. Además, debería generar inmersión y enganche en un jugador.
Por otra parte, se recomiendan ciertos lineamientos como innovación y escalamiento en la dificultad, lo cual permite generar
inmersión (en inglés y en la literatura más conocido como flow), esto le permite a un jugador generar confianza en
un ambiente libre de riesgos para experimentar y poner a prueba sus conocimientos. 

Un problema del aprendizaje activo es que, pese a lograr mejores resultados, incomoda a los estudiantes, pues les
genera la percepción de que su aprendizaje fue menor con respecto a actividades de enseñanza pasiva \cite{active_learning_versus_feeling_of_learning}.
Los autores de este trabajo atribuyen tales resultados al sentimiento de desconocimiento que sienten los alumnos mientras realizan estas actividades,
ahunado a que, al no estar familiarizados con aprendizaje activo intenso en clases, no comprenden que
el esfuerzo cognitivo implicado es un signo de aprendizaje efectivo.

Sin embargo, un videojuego que entregue retroalimentación constante al jugador de que sus acciones
son correctas o incorrectas, sumado a que cada estudiante estaría realizando las actividades
de forma individual, dejaría de exponer sus errores y quitaría esta incomidad para los estudiantes,
sobretodo si el juego logra la inmersión del jugador, haciendo que este olvide que está aprendiendo.

El juego se acotará a los cuatro algoritmos: BFS, DFS, Kruskal y Prim relacionados a grafos a través de distintos niveles.
El jugador podrá elegir partir por BFS o DFS. Una vez los haya ejecutado correctamente, podrá acceder a los niveles para resolver
distancias en grafos con Kruskal y Prim.


% owever, research on using video games to foster CT skills is very limited. A number of
% video games have been developed to teach programming, such as Wu's Castle (Eagle & Barnes, 2009), CodeCombat (https://
% codecombat.com/), CodeSpell (Esper, Foster, Griswold, Herrera, & Snyder, 2014), and MiniColon (Ayman, Sharaf, Ahmed, &
% Abdennadher, 2018). But these games use a text-based programming language that requires the player to pay considerable attention
% to the details of syntax, which is not well aligned with CT




% https://www.incod.ufsc.br/wp-content/uploads/2018/03/Relatorio-Tecnico-INCoD_GQS_05_2018_E.pdf



% Having observed this negative correlation between students’ FOL and their actual learning,
% we sought to understand the causal factors behind this observation. A survey of the 
% existing literature suggests 2 likely factors: 1) the cognitive fluency of lectures can 
% mislead students into thinking that they are learning more than they actually are (30, 31)
%  and 2) novices in a subject have poor metacognition and thus are ill-equipped to judge
%  how much they have learned (27–29). We also propose a third factor: 3) students who are
%  unfamiliar with intense active learning in the college classroom may not appreciate that
%  the increased cognitive struggle accompanying active learning is actually a sign that the
%  learning is effective. We describe below some evidence suggesting that all 3 factors are
%  involved and propose some specific strategies to improve students’ engagement with active
%  learning.



\section{Problema}


Autores como Zehetmeier et al. \cite{abstract_thinking_cs}, Ghezzi et al. \cite{fundamentals_software_abstract} y Wing \cite{wingresearch_abstract} afirman
que el pensamiento abstracto es una de las habilidades fundamentales del pensamiento computacional
y para las ciencias de la computación. Se observa a nivel general que hay estudiantes que ven muy difícil
el abstraer problemas o la forma de una solución. Ante esto, han surgido intentos como Scratch \cite{scratch}, Snap! o 
Blockly \cite{Blockly}, que buscan funcionar como herramientas de programación visuales para facilitar
el aprendizaje y facilitar el proceso de abstracción \cite{programming_education}.
% Scratch (Resnick et al., 2009), Snap! (Harvey & Mönig,
% 2010), and Blockly (Fraser, 2013),

McGonigal \cite{games_makes_us_better} y Barab et al. \cite{barab2009transformational}, indican que los videojuegos son
una herramienta potencial para el aprendizaje y motivación a través de la inmersión en el videojuego. 


\section{Preguntas de Investigación}

P1: ¿Cómo se comparan los resultados en un test de programación entre un grupo de control y otro que
jugó un videojuego que enseña algoritmos relacionados a grafos?

P2: ¿Cómo se comparan los niveles de interés entre un grupo de control y otro que jugó un videojuego
que enseña algoritmos relacionados a grafos?

% 2 cosas: Resultados en nivel de programación y niveles de interés
% los resultados en el nivel de programación son de manera objetiva, se puede validar cuántos llegaron al outcome y cuántos no
% Respecto a los niveles de interés, es más subjetivo. 

\section{Hipótesis}

H1: Un videojuego que enseña algoritmos relacionados a grafos es una herramienta
que mejora los resultados en un test de programación de los estudiantes de computación.

H2: Un videojuego que enseña algoritmos relacionados a grafos genera mayor interés
en el estudiantado.

Se entienden mejores resultados académicos como notas más elevadas en evaluaciones estándares
realizadas por los equipos docentes a cargo del ramo.

Se entiende interés como una medida subjetiva de cuánto un usuario desea
aprender sobre un tema en particular. 

\section{Objetivo principal}

% En la metodología debería especificar lo específico de estos objetivos con las preguntas de investigación
% En la metodología hay que enlazar objetivo con la pregunta de investigación


Diseñar y desarrollar un videojuego que muestre grafos y algoritmos relacionados a estos,
que permita al usuario ir ejecutando acciones en la aplicación según lo pide un algoritmo
desplegado en la misma pantalla. La aplicación debe indicar al usuario cuando está en lo
correcto y cuándo se equivoca.

\section{Objetivos específicos}

\begin{itemize}
\item Diseñar una aplicación interactiva que muestre grafos y permita seguir
los pasos relacionados a algoritmos que trabajen con grafos.

\item Medir el interés de estudiantes de computación relacionado a grafos
antes y después del uso de esta herramienta.

\item Medir la comprensión y capacidad de replicación de un algoritmo relacionado a grafos
de un grupo de estudiantes de computación antes y después de probar el videojuego presentado.

\end{itemize}

\section{Metodología}


Se desarrolla un videojuego considerando el feedback de un profesor guía y otros estudiantes
que no sean parte de la población objetivo del estudio, ajustándolo según sea necesario. \\
Asegurarse que el videojuego cumple con las características seguridas por autores que proponen metodologías para la evaluación de videojuegos educativos: Gibson et al. \cite{evaluation_of_games_for_teaching_cs},
Petri, Giani y otros, \cite{petri2018method} y Kiili et al. \cite{using_videogames_maths}. Si no, seguir iterando. \\
Cuando se haya impartido la materia relacionada a grafos y se hayan visto los algoritmos BFS y DFS, preguntarle a
los alumnos si se creen capaces de programar tales algoritmos desde 0 y guardar estos resultados.
Presentar el juego en los cursos cuando se esté impartiendo la materia relacionada a grafos. El uso del
videojuego será voluntario y sugerido, con el requisito de llenar un formulario posterior al uso del juego.
El formulario poseerá un cuestionario siguiendo los lineamientos de MEEGA+ planteados \cite{petri2018meegaplus} \\
Se revisarán los datos llenados por los estudiantes que accedieron al formulario, en conjunto con los
datos recolectados durante el juego mismo, para ver si estos se condicen. Por ejemplo, el juego medirá el tiempo
entre clicks. Se espera que un jugador que esté motivado posea un alto nivel de actividad, y una alta cadencia de clicks. \\
Finalmente, se realizará una prueba de programación en los cursos que vean la materia de grafos, donde los estudiantes
tendrán que responder preguntas relacionadas a los algoritmos vistos en el juego. Se separarán dos grupos, 
uno de control con la gente que decidió no jugar, y otro experimental que sí probó el videojuego.

El programa ofrecerá a los jugadores ejecutar tareas relacionadas con los algoritmos BFS, DFS, Kruskal y Prim relacionados a grafos.
Para completar los desafíos presentados, el usuario deberá realizar las instrucciones dictadas por un computador, invirtiendo los roles
de un programador y un computador. En este caso, el juego limita al jugador por medio de sus reglas, a seguir los pasos del algoritmo 
presentado. Si el jugador ejecuta correctamente ls instrucciones, ganará, pudiendo pasar a otro nivel.

El videojuego guardará datos durante su ejecución y los enviará a alguna base de datos donde se almacenarán para su posterior
análisis. Los datos que se guardarán serán: clicks, movimientos del mouse, acciones del teclado a lo largo del tiempo.
Estos datos se usarán para contrastar con la información final y para determinar el grado de veracidad de las respuestas.


\subsection{Desventajas y debilidades del proceso}

El tamaño muestral puede no ser suficiente. No se sabe cuánto será el tamaño de los grupos de control
y experimental, pero se espera que el total de participantes sea superior a 50. \\
Existe un sesgo al presentar el uso del videojuego como una opción voluntaria, por lo que los estudiantes
con más tiempo e interés tenderán a probar el videojuego, pero son quienes a su vez tienen mejores notas. \\


\section{Resultados esperados}


Se espera observar un nivel de interés superior en los estudiantes con respecto a tópicos de algoritmos
relacionados a grafos a través del cuestionario entregado.
Se esperan mejores resultados en el grupo que pruebe el videojuego, en un intervalo del
8\% al 14\% de diferencia, lo que debe ser entre 5 y 7 décimas en una escala de evaluación del 1.0 al 7.0
considerando valores decimales intermedios.
Además, se busca generar una metodología de prueba para herramientas interactivas cuyo fin sea la enseñanza,
de tal manera que otro memorista o tesista en el futuro pueda hacer pruebas análogas con otros algoritmos
o estructuras de datos e incluso aplicadas a otras materias.

Por otra parte, se espera publicar un artículo en alguna revista
como ACM o IEEE Education respecto a los resultados obtenidos.


\nocite{*}
\bibliographystyle{eptcs}
\bibliography{generic}
\end{document}
