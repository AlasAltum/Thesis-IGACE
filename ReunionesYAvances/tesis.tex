\documentclass[submission]{eptcs}
\setlength{\parindent}{0pt}
\providecommand{\event}{} % Name of the event you are submitting to

\title{Videojuego educativo para enseñar algoritmos relacionados a grafos}
\author{ Alonso Utreras\\
Profesor Guía: Iván Sipirán
\institute{Departmento de Ciencias de la Computación\\
Universidad de Chile\\
Santiago, Chile}
\email{alonso.utreras@ug.uchile.cl}
\email{autreras@dcc.uchile.cl}
}
\begin{document}
\maketitle

\section{Introducción}


\section{Trabajo relacionado} 



\subsection{Propiedades que debería tener el juego}


% owever, research on using video games to foster CT skills is very limited. A number of
% video games have been developed to teach programming, such as Wu's Castle (Eagle & Barnes, 2009), CodeCombat (https://
% codecombat.com/), CodeSpell (Esper, Foster, Griswold, Herrera, & Snyder, 2014), and MiniColon (Ayman, Sharaf, Ahmed, &
% Abdennadher, 2018). But these games use a text-based programming language that requires the player to pay considerable attention
% to the details of syntax, which is not well aligned with CT




\section{Problema}


\section{Preguntas de Investigación}



\section{Hipótesis}


\section{Objetivo principal}

% En la metodología debería especificar lo específico de estos objetivos con las preguntas de investigación
% En la metodología hay que enlazar objetivo con la pregunta de investigación


Diseñar y desarrollar un videojuego que muestre grafos y algoritmos relacionados a estos,
que permita al usuario ir ejecutando acciones en la aplicación según lo pide un algoritmo
desplegado en la misma pantalla. La aplicación debe indicar al usuario cuando está en lo
correcto y cuándo se equivoca.

\section{Objetivos específicos}

\begin{itemize}
\item Diseñar una aplicación interactiva que muestre grafos y permita seguir
los pasos relacionados a algoritmos que trabajen con grafos.

\item Medir el interés de estudiantes de computación relacionado a grafos
antes y después del uso de esta herramienta.

\item Medir la comprensión y capacidad de replicación de un algoritmo relacionado a grafos
de un grupo de estudiantes de computación antes y después de probar el videojuego presentado.

\end{itemize}

\section{Metodología}


Se desarrolla un videojuego considerando el feedback de un profesor guía y otros estudiantes
que no sean parte de la población objetivo del estudio, ajustándolo según sea necesario. \\
Asegurarse que el videojuego cumple con las características seguridas por autores que proponen
metodologías para la evaluación de videojuegos educativos: Gibson et al. \cite{evaluation_of_games_for_teaching_cs},
Petri, Giano y otros, \cite{petri2018method} y Kiili et al. \cite{using_videogames_maths}. Si no, seguir iterando. \\
Cuando se haya impartido la materia relacionada a grafos y se hayan visto los algoritmos BFS y DFS, preguntarle a
los alumnos si se creen capaces de programar tales algoritmos desde 0 y guardar estos resultados.
Presentar el juego en los cursos cuando se esté impartiendo la materia relacionada a grafos. El uso del
videojuego será voluntario y sugerido, con el requisito de llenar un formulario posterior al uso del juego.
El formulario poseerá un cuestionario siguiendo los lineamientos de MEEGA+ planteados \cite{petri2018meegaplus} \\
Se revisarán los datos llenados por los estudiantes que accedieron al formulario, en conjunto con los
datos recolectados durante el juego mismo, para ver si estos se condicen. Por ejemplo, el juego medirá el tiempo
entre clicks. Se espera que un jugador que esté motivado posea un alto nivel de actividad, y una alta cadencia de clicks. \\
Finalmente, se realizará una prueba de programación en los cursos que vean la materia de grafos, donde los estudiantes
tendrán que responder preguntas relacionadas a los algoritmos vistos en el juego. Se separarán dos grupos, 
uno de control con la gente que decidió no jugar, y otro experimental que sí probó el videojuego.

El programa ofrecerá a los jugadores ejecutar tareas relacionadas con los algoritmos BFS, DFS, Kruskal y Prim relacionados a grafos.
Para completar los desafíos presentados, el usuario deberá realizar las instrucciones dictadas por un computador, invirtiendo los roles
de un programador y un computador. En este caso, el juego limita al jugador por medio de sus reglas, a seguir los pasos del algoritmo 
presentado. Si el jugador ejecuta correctamente ls instrucciones, ganará, pudiendo pasar a otro nivel.

El videojuego guardará datos durante su ejecución y los enviará a alguna base de datos donde se almacenarán para su posterior
análisis. Los datos que se guardarán serán: clicks, movimientos del mouse, acciones del teclado a lo largo del tiempo.
Estos datos se usarán para contrastar con la información final y para determinar el grado de veracidad de las respuestas.


\subsection{Desventajas y debilidades del proceso}

El tamaño muestral puede no ser suficiente. No se sabe cuánto será el tamaño de los grupos de control
y experimental, pero se espera que el total de participantes sea superior a 50. \\
Existe un sesgo al presentar el uso del videojuego como una opción voluntaria, por lo que los estudiantes
con más tiempo e interés tenderán a probar el videojuego, pero son quienes a su vez tienen mejores notas. \\


\section{Resultados esperados}


\section{Conclusiones}

\subsection{Trabajo realizado}


\subsection{Discusión}


\subsection{Trabajo futuro}
* 




\nocite{*}
\bibliographystyle{eptcs}
\bibliography{generic}
\end{document}
