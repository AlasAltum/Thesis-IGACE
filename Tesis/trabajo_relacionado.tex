\chapter{Trabajo Relacionado}


\subsection{Videojuegos educativos o serios}

Un videojuego representa una modalidad de aprendizaje activo, donde el proceso de enseñanza no sigue la estructura tradicional que consiste en tener a un profesor exponiendo frente a un estudiante. En este enfoque, el aprendizaje implica la ejecución de pasos y la participación activa del estudiante en actividades que contribuyen a la construcción del conocimiento. La revisión realizada por Hartikainen et al. \cite{active_learning_review} presenta argumentos a favor del aprendizaje activo, destacando mejores resultados, respaldo político y las demandas contemporáneas del entorno laboral, como habilidades de comunicación y la capacidad de aprendizaje autónomo.

Los videojuegos serios, también conocidos como ``Serious Gaming'', constituyen una forma de aprendizaje activo. Bell y Gibson \cite{evaluation_of_games_for_teaching_cs} sostienen que los juegos educativos son más efectivos que las clases tradicionales, lecturas, videos y tareas. Estos autores afirman que el uso de juegos conlleva a una mejora en la retención, conocimiento factual, habilidades basadas en el conocimiento, y autoeficacia. No obstante, recomiendan que los juegos deben complementarse con otras formas de enseñanza y actividades posteriores al juego que permitan a los estudiantes reflexionar sobre la relación entre los juegos y la materia.

\subsection{Análisis de otros autores sobre el potencial y falencias al trabajar con videojuegos educativos}

Bell y Gibson \cite{evaluation_of_games_for_teaching_cs} identificaron y clasificaron 41 videojuegos de Ciencias de la Computación (CS). Uno de ellos, ``Map Coloring", aborda el tema de grafos, en particular el coloreo de grafos. Aunque se realizó una búsqueda del juego hasta la fecha (2022), no se encontró material al respecto.

Kiili y su equipo \cite{using_videogames_maths} analizaron el uso de videojuegos en enseñanza y evaluación en matemáticas a través de los títulos ``Semideus'' y "Wuzzit Trouble". A través de estos, llegaron a la conclusión de que es posible utilizar videojuegos para enseñar y evaluar al mismo tiempo. Además, caracterizaron estadísticamente las diferencias producidas por el uso de videojuegos entre resultados de un pre test y un post test.

% En otro trabajo de Kiili et. al \cite{Kiili_game_based_rational} realizaron una regresión lineal entre dos variables, puntaje obtenido en el videojuego y mejoras en una prueba académica relacionada con la materia enseñada. En este caso se enseñaba sobre fracciones y números racionales. En este caso, el tiempo dedicado por los grupos de estudio, fue un total de 2 horas y media. % TODO: revisar mejor la relación entre las variables.
% @article{Kiili_game_based_rational,
%   title = {Evaluating the effectiveness of a game-based rational number training - In-game metrics as learning indicators},
%   author = {Kristian Kiili and Korbinian Moeller and Manuel Ninaus},
%   journal = {Computers & Education},
%   volume = {120},
%   pages = {13-28},
%   year = {2018},
%   doi = {https://doi.org/10.1016/j.compedu.2018.01.012},
%   url = {https://www.sciencedirect.com/science/article/pii/S0360131518300125}
% }

En el trabajo de Zhao y Shute \cite{video_game_foster_computational_thinking}, se enlistan ejemplos de videojuegos diseñados para enseñar programación, como Wu's Castle \cite{wuscastle}, CodeCombat \cite{CodeCombat}, CodeSpell \cite{codespells}, y MiniColon \cite{minicolon}. Estos ejemplos emplean programación con texto. No obstante, también existen numerosos referentes que utilizan programación por bloques, como LightBot \cite{LightBot}, Scratch \cite{ scratch, maloney2010scratch}, y RoboBuilder \cite{RoboBuilder}, los cuales simplifican el proceso de aprender la sintaxis relacionada con los lenguajes de programación.

A pesar de esto, el impacto de estos videojuegos no ha sido evaluado en muchos casos. En las instancias documentadas, las muestras suelen ser reducidas, y las evaluaciones se centran principalmente en aspectos cualitativos \cite{video_game_foster_computational_thinking, effectiveness_gbl}.

Petri y otros \cite{meegaplus} coinciden en la falta de sistematización en la evaluación de videojuegos educativos. De hecho, existen juegos serios que ni siquiera se autodenominan o se consideran como tales \cite{evaluation_of_games_for_teaching_cs}. Además, muchos estudios eran cualitativos y carecían de resultados cuantitativos, por lo que no eran comparables.En vista de este antecedente, Petri et al. \cite{meegaplus} desarrollaron el modelo MEEGA+ (Modelo para la Evaluación de Juegos Educativos y Escala EGameFlow).

Entre las deficiencias señaladas al momento de crear videojuegos según Petri et al. \cite{meegaplus} se encuentran la carencia de: 1) Definición de un objetivo de evaluación; 2) Diseño de investigación; 3) Plan de medición; 4) Instrumentos de recopilación de datos; y 5) Métodos de análisis de datos. Una práctica que era común al analizar estas herramientas es hacer uso de comentarios informales por parte del estudiantado, sin la aplicación de escalas psicométricas \cite{meegaplus}.
