\chapter*{Anexo F: Comentarios abiertos entregados por los voluntarios}\label{AnexoF}

\newgeometry{margin=0.5in}

% 1.1

\begin{table}[h]
   \centering
   \caption*{\textit{Comentarios del grupo que realizó la prueba económica frente a la pregunta: Nombra aspectos fuertes del juego}}
   \begin{tabular}{|p{\linewidth}|}
   \hline
   \textbf{Comentario} \\\hline
   Es una forma distinta de aprender. Es interactivo. Me gusta que sea de prueba y error. \\\hline
   Enseña de manera correcta el algoritmo \\\hline
   Era intuitivo y fácil de entender \\\hline
   El aspecto que me llamó mucho la atención y lo destaco como un aspecto fuerte del juego, fue la forma interactiva de mostrar un código de programación y conceptos detrás de estos mismos. \\\hline
   
   De esa forma, se puede visualizar de mejor manera la dinámica detrás de un algoritmo (que muchas veces eso llega a ser un problema a la hora de analizar su funcionamiento, en especial si son algoritmos más complejos) \\\hline
   Me parece que al ser interactivo se entienden mejor los conceptos, además los movimientos de la nave ayudan a entender gráficamente cómo funciona el algoritmo \\\hline
   El diseño y la historia planteada son atractivos, como también la jugabilidad, es idónea para adquirir habilidades de programación. \\\hline
   
   Es intuitivo y no tan difícil de entender la idea, el diseño de niveles y la interfaz de usuario \\\hline
   Incorporar animales tiernos y buena música aporta mucho a la satisfacción y motivación de terminar los niveles \\\hline
   El acompañamiento visual de las líneas del código ayuda mucho al entendimiento del algoritmo sin tener la necesidad de explicar textualmente la función de cada uno \\\hline
 Me gustó poder saltarme las instrucciones porque no me gusta leerlas, menos si es un juego. Me gustaron los colores, llama la atención a pesar de ser un juego sin mucho detalle \\\hline
    La atmósfera del juego permite conectar fácilmente con la experiencia de aprendizaje, además su temática me parece bastante atractiva \\\hline
    Es entretenido y super fácil de seguir las instrucciones \\\hline
    Me gustó que agregue una imagen sobre el contenido de programación que se tratará en cada nivel, eso me facilita el aprendizaje y la idea de lo que estoy realizando. Además, el juego ayuda bastante si te equivocas o no realizaste una acción (como seleccionar el nodo u otros) \\\hline
    Es interesante que sea de programación y astronomía/ es desafiante \\\hline
    Es un juego entretenido, los gráficos llaman la atención, interactivo y uno puede quedarse pegado jugando \\\hline
    Las gráficas son muy bonitas y atractivas, invitan a jugar  \\
   \hline
   \end{tabular}
\end{table}

\restoregeometry

% 1.2

\newgeometry{margin=0.5in}
\begin{table}[h]
   \centering
   \caption*{\textit{Comentarios a la pregunta ``indica una o más sugerencias para mejorar el juego''. Grupo que realizó la prueba académica.}}
   \begin{tabular}{|p{\linewidth}|}
   \hline
   \textbf{Comentario} \\\hline
   Mejorar colocación de títulos. \\ \hline
   Hacerlo menos tedioso, quizás agregar alguna música acorde, poner el código de manera más amigable en vez de un código tan seco, que el grafo inicial sea más pequeño; \\ \hline
   Quizás se podría explicar un poco más de qué significa el código de la derecha para gente que no tiene ni idea de programación. Que es un for, un while y demás, explicar cómo se recorre y por qué; \\ \hline
   Agregar muchos más niveles que formen una gran escalera progresiva de contenidos a aprender. También podría sugerir que en el momento que la persona no interactúe con el juego en un tal momento del algoritmo, el juego diga mini hints que ayuden a pensar a la persona de lo que debería hacer en donde está estancado. Tal vez que sea más largo o que tenga más ejercicios por algoritmo; \\ \hline
   La tipografía de la letra y los colores para facilitar la lectura, hay números que se confunden con letras y requieren contraste con el resto del juego, un buen ejemplo de esto sería el final fantasy donde el texto es solamente un fondo negro con letras blancas los controles, al momento de agregar los planetas yo pensaba que era apretar r no más pero había que poner el mouse sobre el planeta cambiar la letra no me gustaba; \\ \hline
   Las palabras del texto explicativo deberían resaltar las instrucciones centrales. \\ \hline
   Sería positivo incorporar un botón de ayuda, en algunas partes sentí que me quedé atrapado.; \\ \hline
   Las instrucciones del principio deberían ser entregadas de una manera más simple y directa. Quizás un cuadro con una lista de todo lo que hay que hacer al inicio podría ayudar.; \\ \hline
   Mejorar la fuente de texto, se me complicó en ciertos casos leer algunas letras.; \\ \hline
   Podría ser la música que era un poco desesperante y se podría hacer un pequeño tutorial explicando las funciones que se utilizan; \\ \hline
   Si el juego busca enseñar programación, sería bueno colocar un módulo donde se pueda explicar mejor el código o dar alguna explicación de ciertas funciones que no sabía qué hacían pero igual tenía que apretar espacio para pasarlas (podría ser una pequeña explicación al pasar el mouse por sobre el texto). Quizás no era de importancia para el juego, pero como no conozco mucho el lenguaje con el que se utilizó e igual sé programar, sería interesante poder conocer qué significan tales funciones.; \\ \hline
   Encuentro que el juego se puede completar sin entender del todo la diferencia entre queues y stacks, solo mecanizando el seguir las instrucciones del panel derecho. Ambos niveles se diferencian en que el primero es mucho más rápido de recorrer y mecanizar, mientras que el segundo plantea más dificultad, pero no queda completamente claro en cuál se usan queues y en cuál stacks.; \\ \hline
   Quizás faltaron planetas para que se notara más el hecho de acumular elementos en queues o stacks y la forma en que estos se extraen, o quizás sería más sencillo entender estos conceptos si el juego solo consistiera en clickear planetas para ir recorriendo los mapas, sin necesidad de apretar teclas como espacio, R, W o S, ya que a la larga esta mecánica distrae un poco de entender realmente cómo están operando los algoritmos. \\ \hline
   \end{tabular}
\end{table}

\restoregeometry

% 1.3
\newgeometry{margin=0.5in}
\begin{table}[h]
   \centering
   \caption*{\textit{Comentarios a la pregunta ``Algún otro comentario?''. Grupo que realizó la prueba académica.}}
   \begin{tabular}{|p{\linewidth}|}
   \hline
   \textbf{Comentario} \\\hline
   Muy buen juego, es a mi gusta una forma divertida de aprender a manejar grafos. \\ \hline
   Muy entretenido, viva los pandas rojos!! \\ \hline
   Muy interesante y entretenida la propuesta de este juego. Espero que se logré desarrollar más y más :3 \\ \hline
   No soy fan del tema espacial pero si me gusto el juego \\ \hline
   muy bueno \\ \hline
   Las palabras del texto explicativo deberían resaltar las instrucciones centrales. \\ \hline
   Hay unos pasos en los que no hay que hacer ninguna acción, por ejemplo el comando s.pop, y al principio eso me confundió un poco por no tener la certeza de cuál era su función dentro el algoritmo. Creo que eso es parte del desafío del juego y se logra comprender gracias al acompañamiento visual y sonoro \\ \hline
   creo que a pesar de ser estudiante de ingeniería  tengo poca noción de la programación, así que para mi al no leer las instrucciones necesite ayuda para terminar el juego. Ayudaría quizás en el lado derecho no poner un código y si una explicación con palabras \\ \hline
   En líneas generales una excelente temática y experiencia. \\ \hline
   Buen juego, entretenido y didactico para aprender la materia de programación. Me gustaria que hubiesen más juegos así para abordar otros contenidos de programación. \\ \hline
   Muy bueno el juego!!! Felicitaciones \\ \hline
   aguanten los pandas! \\ \hline
   \end{tabular}
\end{table}

\restoregeometry





\section*{Comentarios del grupo que participó en el estudio libre sin restricciones}


\newgeometry{margin=0.5in}

\begin{table}[h]
   \centering
   \caption*{\textit{Comentarios a la pregunta: ``Por favor, indica uno o más aspectos fuertes del juego'' Grupo libre.}}
   \begin{tabular}{|p{\linewidth}|}
   \hline % Linea horizontal inicial
   1. El ir paso a paso ejecutando los algoritmos ayuda mucho a entender cómo funcionan y cómo se diferencian entre sí. 2. El feedback visual de algunas cosas como la variable seleccionada o el estado actual del stack/queue que se está usando permite siempre saber lo que está ocurriendo, o incluso retomar la ejecución luego de perder la atención un momento. \\\hline

   La musica es bien llamativa y la tematica de la nave buscando los pandas rojos es atractiva. \\\hline

   La idea de representar lo importante de programar gráficamente es muy interesante, la idea de abstraerlo a una temática de exploración espacial también lo es. \\\hline
   
   Me gusta que te obligue a hacer las instrucciones una a una. \\\hline

   El aspecto visual es bueno, la forma retro del código mostrado al lado derecho me gusta. \\\hline
   \hline
   \end{tabular}
\end{table}

\restoregeometry


\newgeometry{margin=0.5in}

\begin{table}[h]
   \centering
   \caption*{\textit{Comentarios a la pregunta ``indica una o más sugerencias para mejorar el juego''. Grupo libre.}}
   \begin{tabular}{|p{\linewidth}|}
   \hline % Linea horizontal inicial
   1. El juego se beneficiaría de más feedback visual sobre las cosas seleccionadas/actuales en el algoritmo, como cuales son los nodos vecinos.
   2. Además, un nivel inicial con nodos ordenados de forma que las líneas no se sobrepongan puede ser útil.
   3. Para notar la diferencia entre BFS y DFS, tener una misma configuración de nodos puede ayudar a entender cómo ocurre que ciertos planetas se visitan antes que otros según el algoritmo que se esté usando.
   4. Finalmente, animaciones in-game para explicar los algoritmos pueden funcionar mucho mejor que gifs a pantalla completa. \\\hline


   Algunos de los gifs de fondo que muestran el recorrido de grafos podrían tener mejor calidad, y creo que algunos dialogos se muestran muy lento en comparacion a otros.
   creo que hace falta niveles más profundos en los algoritmos de recorrido de grafos, para entender una dinámica mayor y quizás algún nivel que te dejen solo y por los requisitos de algún tipo, tengas que utilizar alguno de los algoritmos enseñados\\\hline

   Considero que una mejora en la selección de colores y distribuciones espaciales de los elementos en pantalla haría que gráficamente el juego mejorara mucho, y pienso que eso a su vez ayudaría a que fuera mucho más interesante aprender mediante él. Por otro lado, creo que se le debería dar más libertad al jugador, en especial la posibilidad de que se pudiera equivocar y este error afecte en el funcionamiento del programa (explicando gráficamente de alguna manera por qué se equivocó y qué consecuencias tiene eso), entiendo que es algo complicado de implementar pero creo que si es implementado de una manera inteligente puede potenciar mucho el aprendizaje.\\\hline

   Algunas cosas. Hay un bug que cuando aparece una de las imágenes de pandas rojo, el espacio deja de funcionar para avanzar. \\\hline 
   
   Creo que el no explicar con anterioridad las instrucciones es un poco complicado. Me ví haciendo cosas que no entendía y simplemente siguiendo las instrucciones sin ver el algoritmo a gran escala, además estaba camuflado con todos los aspectos del juego. Creo que sería bueno poner un paralelo un poco más formal a medida que se desarrolla el juego, quizá más animaciones para mostrar que el planeta que seleccionas efectivamente se va a la cola, y quizá además del número debería haber una miniatura del planeta en la lista.  \\\hline
   
   Otra cosa que me pasó un par de veces es que sin querer apreté la barra espaciadora y por coincidencia estaba en la opción correcta, por lo que no entendí qué hice y el juego avanzó. Quizá en cada iteración la consola debería reinicial la posición para tener que mover el selector de manera voluntaria y entender el algoritmo. \\\hline
   
   Otro punto es que cuando haces algo bien, hay un sonio que uno termina relacionando con haber hecho bien el paso en el algoritmo. En algún momento el sonido deja de sonar al inicio de los for, lo que es raro porque no sabes si lo hiciste bien o no. \\\hline

   El aspecto de la cola, ver la forma de avanzar más rápido en esos pasos estaría bueno. \\
   \hline
   \end{tabular}
\end{table}

\restoregeometry


\newgeometry{margin=0.5in}

\begin{table}[h]
   \centering
   \caption*{\textit{Comentarios a la pregunta ``Algún otro comentario?''. Grupo libre.}}
   \begin{tabular}{|p{\linewidth}|}
   \hline % Linea horizontal inicial
   \textbf{Comentario} \\\hline
   Es un juego completamente necesario y una buena idea que puede facilitar mucho el aprendizaje de grafos. Gran trabajo
   PD: Tal vez tener alguna forma de configurar niveles custom podría hacer de esta una herramienta de debugging, que puede estar muy bien :D \\\hline
   Me parecio bien entretenido en general \\\hline
   
   Considero admirable el aportar en el aprendizaje de la computación mediante juegos, y muy valiente considerando que no es una opción "popular" con mucha información al respecto \\\hline
   Me gusta la idea pero creo que hay que pulir los controles y las instrucciones \\\hline
   Me gustó el concepto. Se entiende bien el concepto de búsqueda en grafo y queda ``grabado'' al intentarlo varias veces. Eso sí, como mi objetivo era avanzar en el juego lo hice pese a que los ejercicios con la cola a veces eran más complejos en el sentido de avanzar al estar atento a pulsar el botón "espacio", pero en general me gustó el juego y cómo se aprende en el proceso.
   \\\hline
   \end{tabular}
\end{table}

\restoregeometry

