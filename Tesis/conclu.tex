\chapter{Conclusión}

El videojuego logra su objetivo de enseñar los algoritmos de BFS y DFS, como evidencian los resultados obtenidos por estudiantes de informática de la Universidad de Chile. Estos estudiantes demostraron comprender los algoritmos, así como la naturaleza de los grafos, al responder de manera acertada a una prueba asociada al juego.

Por otra parte, el juego ha recibido una evaluación positiva y se considera, según el modelo MEEGA+ \cite{meegaplus}, como un juego de calidad. A pesar de las mejoras identificadas durante el proceso, el juego ha sido bien recibido por los estudiantes, y su percepción positiva puede influir beneficiosamente en la motivación y, por ende, en los resultados académicos.

Las hipótesis son verdaderas. La primera hipótesis indica que un videojuego que enseña sobre grafos puede enseñar a los estudiantes sobre los algoritmos BFS y DFS. La segunda hipótesis, \emph{H2 Un videojuego que enseñe grafos será percibido de manera positiva por los estudiantes que todavía no aprenden sobre esos contenidos}, también es correcta, medida a través del valor $\theta$ entregado por el modelo MEEGA+.

Este trabajo sienta bases para futuras investigaciones, especialmente en relación con el diseño de videojuegos educativos y su impacto en la educación en informática. La metodología utilizada aquí puede ser perfeccionada en trabajos futuros para evaluar mejor la comprensión de los conceptos enseñados mediante este enfoque lúdico.

Para casos futuros, se recomienda emplear una metodología de estudio pre/post/post como se indica en \cite{HowGamesComputingEducationEvaluated}, aumentar el tamaño muestral y utilizar al menos dos grupos: uno de control y otro de tratamiento para medir mejor las diferencias. 



% Énfasis en Resultados Positivos: Resalta de manera más específica los resultados positivos obtenidos, como el hecho de que los estudiantes lograron comprender los algoritmos y la estructura de los grafos, así como aprobar la prueba asociada. Esto ayuda a consolidar la efectividad del videojuego como herramienta educativa.

% Impacto en la Motivación: Puedes profundizar en cómo la percepción positiva del juego podría influir en la motivación de los estudiantes. ¿Se notó un aumento en el interés por el tema? ¿Hubo comentarios que sugieran un cambio en la actitud hacia el aprendizaje?

% Relación con Otros Métodos de Enseñanza: Considera mencionar cómo este enfoque se integra o complementa con otros métodos de enseñanza tradicionales. ¿El juego podría servir como apoyo en determinadas etapas del aprendizaje?

% Desafíos y Lecciones Aprendidas: Brevemente, comparte los desafíos encontrados durante el desarrollo e implementación del videojuego, así como las lecciones aprendidas. Esto podría incluir aspectos técnicos, metodológicos o de diseño que podrían ser útiles para otros investigadores o desarrolladores en proyectos similares.

% Relevancia en la Educación Actual: Haz hincapié en la relevancia del juego en el contexto educativo actual. ¿Cómo se alinea con las tendencias y necesidades de la educación en informática?

% Posibles Extensiones del Trabajo: Además de lo que ya has mencionado sobre futuras investigaciones, podrías sugerir posibles extensiones o mejoras específicas para el juego en función de los comentarios y resultados obtenidos.