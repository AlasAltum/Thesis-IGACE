\chapter*{Anexo G: Comentarios de algunos expertos}\label{AnexoG}

Estos comentarios han sido anonimizados y aleatoriazados para proteger la identidad de quienes los emitieron, pues se les garantizó anonimato en caso de que sus comentarios se publicaran.
Muchos comentarios requierne más contexto para ser entendidos, puesto que se enmarcaban en medio de una prueba de usuario, frente a cierta interfaz. Por esta razón, se hizo una selección de comentarios y se muestran aquí.


\newgeometry{margin=0.5in}

\begin{table}[h]
   \centering
   \caption*{\textit{ }}
   \begin{tabular}{|p{\linewidth}|}
   \hline
   \textbf{Comentario} \\\hline
   El juego debería ser consistente, si quieres avanzar con SPACE, apreta SPACE para continuar. Siempre. \\\hline
   (Después de presionar un planeta, hace click en uno y no pasa nada) ¿Qué pasa al hacer click en un planeta? mmm, nada. Falta feedback ahí. \\\hline
   
   En un juego siempre hay que tener clara la condición de ganar.
   Cuál es el objetivo para ganar? La misión es siempre la misma? Anda subdividiendo la misión. \\\hline

   Te recomiendo hacer tutoriales de a poco, como Duolingo. AL principio te muestran las mecánicas con ejercicios muy simples, y luego te dejan andar. \\\hline

   Haz tutoriales, es como la técnica del triciclo. Al principio los ayudas, después los dejas andar solos \\\hline

   No recomiendo destacar las cosas solo con color. Agrega movimiento también. Para la gente con daltonismo puede ser complejo. \\\hline

   Trata de que el mouse cambie cuando pasas por un elemento clickeable o seleccionable. \\\hline

   (Respecto a la popup con un if) La fuente del Yes y No está re mala. Parece muy poco real. \\\hline

   Usa una única fuente constante de información. Si me quedo con una, que sea solo una popup. \\\hline

   Aprovecha el movimiento. Las cosas que se mueven llaman mucho más la atención. Si quieres que hagan click en algo, agrégale movimiento. Si quieres que el usuario lea algo, agrégale movimiento \\\hline

   Prueba testear conceptos de a poquito: Quiero testear este tutorial. Qué cosas ven primero? Después, prueba cambiar los colores de los planetas, resaltarlos, vé cómo eso afecta. Pero es importante probar paso por paso, si haces todos los cambios de una sin probar y falla algo, perderás mucho tiempo y no sabrás exactamente qué falló. \\\hline

   Busca conocer bien a tu público objetivo. Si son gente que está dado Algoritmos y Estructuras de Datos, ve qué edad tienen, qué tipo de animaciones llaman más su atención (...). \\\hline

   El núcleo del juego se debe tratar de que estás siguiendo las instrucciones (...). Lo ideal es no explicitarlo y que se dé a entender por sí solo. \\\hline

   
   Confía en el conocimiento de tu usuario. Gente universitaria que juega
   juegos. Selecciona este nodo y agrégalo a la variable. Que el usuario
   descubra a través de la interfaz cómo se hace eso
   Si la interfaz está bien hecha, el usuario debería encontrar cómo hacerlo. \\\hline

   Hay un concepto que se llama la ceguera del cambio.
   Uno no detecta los cambios en los que no está enfocado
   Si estoy enfocado en cierta parte de la pantalla, hay una parte que está cambiando y no la voy a ver si me pierdo esta información.
   Enfocar la atención del usuario en un solo lado, que es donde estoy dando la instrucción, procura que las instrucciones que se den con una sola forma. \\\hline

   Podrías simplificar mucho más el primer acercamiento. Tutoriales interactivos de lenguajes de programación: Esto es un if, lo que está dentro, así conectas lo que hiciste en el primer, paso con un conocimiento nuevo en el segundo paso \\\hline

   Algo importante en UX/UI es nunca sorprender al usuario. Mientras menos tenga que aprender, mejor. Lo ideal es usar estándares y colgarde de ellos. Piensa en la Ley de Jacob: Los usuarios gastan la mayor parte de su tiempo en otros sitios. Tu juego representa un porcentaje enano en la vida de los demás usuario.  \\

   \hline
   \end{tabular}
\end{table}

\restoregeometry
