\chapter*{Anexo A: Formulario basado en MEEGA+} 

\label{AnexoA}

\newgeometry{margin=0.5in}

\begin{table}[h]
\centering
\caption{Formulario basado en MEEGA+. Se eliminó la dimensión social y no se incluyeron los ítems que preguntan por los objetivos particulares del juego.}
\label{TablaFormularioMEEGA}
\begin{tabular}{|c|l|} % borde | columna fuente centrada | borde | columna fuente a la izq | 
\hline % Linea horizontal inicial
\textbf{Notación} & \textbf{Pregunta} \\\hline % Separa el inicio
Q1       & El diseño de juego es atractivo \\ 
Q2       & La fuente de texto y los colores están bien combinados y son consistentes \\
Q3       & Tuve que aprender cosas antes de poder jugar \\
Q4       & Aprender a jugar se me hizo fácil \\
Q5*      & Creo que la mayoría de la gente aprendería a usar este juego rápidamente. \\
Q6       & Creo que el juego es fácil de jugar. \\
Q7       & Las reglas del juego son claras y fáciles de entender \\
Q8       & Las fuentes (su tamaño y estilo) usadas son fáciles de leer \\
Q9       & Los colores usados en el juego son significativos \\
Q10*     & El juego permite customizar la apariencia (fuentes y color) según mis preferencias \\
Q11*     & El juego previene que cometa errores \\
Q12*     & Cuando cometo un error, es fácil recuperarse de él \\
Q13      & Cuando vi el juego por primera vez, tuve la sensación de que sería fácil para mí \\
Q14      & La estructura me ayudó a tener confianza en que aprendería con este juego \\
Q15      & Este juego es desafiante en la medida justa \\
Q16      & El juego ofrece nuevos desafíos a un ritmo adecuado \\
Q17      & El juego no se vuelve monótono a medida que se progresa \\
Q18      & Completar las tareas del juego me provocó satisfacción \\
Q19      & Gracias a mi esfuerzo personal pude avanzar en el juego \\
Q20      & Siento satisfacción con lo aprendido en este juego \\
Q21      & Recomendaría este juego a mis colegas \\
Q22*     & Pude interactuar con otros jugadores mientras jugaba. \\
Q23*     & El juego promueve la cooperación o competencia entre jugadores. \\
Q24*     & Me sentí bien interactuando con otros jugadores durante el juego. \\
Q25      & Me entretuve con el juego \\
Q26      & Algún elemento del juego me hizo sonreír \\
Q27      & Había algo interesante al inicio del juego que llamó mi atención \\
Q28      & Estaba tan envuelt@ en la tarea propuesta por el juego que perdí la noción del tiempo \\
Q29      & Me olvidé de mi entorno físico mientras jugaba \\
Q30      & Los contenidos del juego son relevantes para mis intereses \\
Q31      & Es claro ver que los contenidos del juego se relacionan a cierta materia de la carrera \\
Q32      & Este juego es adecuado para enseñar el contenido \\
Q33      & Prefiero aprender con este juego en vez de aprender con otros métodos de enseñanza \\
Q34      & El juego contribuyó a mi aprendizaje \\
Q35      & El juego me permitió aprender de forma eficiente en comparación con otras actividades \\
& \\

\hline
\end{tabular}
\vspace{0.5em} % Adjust the vertical space
\par
Preguntas con * es porque no estaban en el formulario. Se omitieron y se rellenaron con 0 porque el juego no incluía interacción con otros jugadores ni permitía customización.

\end{table}

\restoregeometry
