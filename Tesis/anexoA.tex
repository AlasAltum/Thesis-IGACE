\chapter{Anexo A: Formulario basado en MEEGA+} \label{AnexoA}

\newgeometry{margin=0.5in}

\begin{table}[h]
\centering
\caption{Formulario basado en MEEGA+. Se eliminó la dimensión social y no se incluyeron los ítems que preguntan por los objetivos particulares del juego.}
\label{TablaFormularioMEEGA}

% i1*: El diseño de juego es atractivo
% i2: La fuente de texto y los colores están bien combinados y son consistentes
% i3: Tuve que aprender cosas antes de poder jugar
% i4: Aprender a jugar se me hizo fácil 
% i5*: Creo que la mayoría de la gente aprendería a usar este juego rápidamente.
% i6*: Creo que el juego es fácil de jugar.
% i7*: Las reglas del juego son claras y fáciles de entender
% i8: Las fuentes (su tamaño y estilo) usadas son fáciles de leer
% i9: Los colores usados en el juego son significativos
% i10: La estructura me ayudó a tener confianza en que aprendería con este juego
% i11: Este juego es desafiante en la medida justa
% i12: El juego ofrece nuevos desafíos a un ritmo adecuado
% (i13 manual): El juego no se vuelve monótomo a medida que se avanza
% (i13 en paper): Cuando vi el juego por primera vez, tuve la sensación de que sería fácil para mí
% i14*: Completar las tareas del juego me provocó satisfacción
% i15*: Pude avanzar en el juego gracias a mi esfuerzo personal
% i16*: Siento satisfacción con lo aprendido en este juego
% i17*: Recomendaría este juego a mis colegas

% *** Social questions are skipped, from i18 to i20.
% *** Pude interactuar con otros jugadores mientras jugaba.
% *** El juego promueve la cooperación o competencia entre jugadores.
% *** Me sentí bien interactuando con otros jugadores durante el juego.

% i21*: Me entretuve con el juego
% i22*: Algún elemento del juego me hizo sonreír
% i23*: Había algo interesante al inicio del juego que llamó mi atención
% i24*: Estaba tan envuelt@ en la tarea propuesta por el juego que perdí la noción del tiempo
% i25*: Me olvidé de mi entorno físico mientras jugaba
% i26*: Los contenidos del juego son relevantes para mis intereses
% i27*: Es claro ver que los contenidos del juego se relacionan a cierta materia de la carrera
% i28*: Este juego es adecuado para enseñar el contenido
% i29*: Prefiero aprender con este juego en vez de aprender con otros métodos de enseñanza
% i30*: El juego contribuyó a mi aprendizaje
% i31*: El juego me permitió aprender de forma eficiente en comparación con otras actividades


\begin{tabular}{|c|l|} % borde | columna fuente centrada | borde | columna fuente a la izq | 
\hline % Linea horizontal inicial
\textbf{Notación} & \textbf{Pregunta} \\\hline % Separa el inicio
Q1       & El diseño de juego es atractivo \\ 
Q2       & La fuente de texto y los colores están bien combinados y son consistentes \\
Q3       & Tuve que aprender cosas antes de poder jugar \\
Q4       & Aprender a jugar se me hizo fácil \\
Q5       & Creo que la mayoría de la gente aprendería a usar este juego rápidamente. \\
Q6       & Creo que el juego es fácil de jugar. \\
Q7       & Las reglas del juego son claras y fáciles de entender \\
Q8       & Las fuentes (su tamaño y estilo) usadas son fáciles de leer \\
Q9       & Los colores usados en el juego son significativos \\
Q10      & La estructura me ayudó a tener confianza en que aprendería con este juego \\
Q11      & Este juego es desafiante en la medida justa \\
Q12      & El juego ofrece nuevos desafíos a un ritmo adecuado \\
Q13      & Cuando vi el juego por primera vez, tuve la sensación de que sería fácil para mí \\
Q14      & Completar las tareas del juego me provocó satisfacción \\
Q15      & Siento satisfacción con lo aprendido en este juego \\
Q16      & Siento satisfacción con lo aprendido en este juego \\
Q17      & Recomendaría este juego a mis colegas \\
Q18*     & Pude interactuar con otros jugadores mientras jugaba. \\
Q19*     & El juego promueve la cooperación o competencia entre jugadores. \\
Q20*     & Me sentí bien interactuando con otros jugadores durante el juego. \\
Q21*     & Me entretuve con el juego \\
Q22      & Algún elemento del juego me hizo sonreír \\
Q23      & Había algo interesante al inicio del juego que llamó mi atención \\
Q24      & Estaba tan envuelta/o en la tarea propuesta por el juego que perdí la noción del tiempo \\
Q25      & Me olvidé de mi entorno físico mientras jugaba \\
Q26      & Los contenidos del juego son relevantes para mis intereses \\
Q27      & Es claro ver que los contenidos del juego se relacionan a cierta materia de la carrera \\
Q28      & Este juego es adecuado para enseñar el contenido \\
Q29      & Prefiero aprender con este juego en vez de aprender con otros métodos de enseñanza \\
Q30      & El juego contribuyó a mi aprendizaje \\
Q31      & El juego me permitió aprender de forma eficiente en comparación con otras actividades \\
& \\

\hline
\end{tabular}
\end{table}

\restoregeometry
