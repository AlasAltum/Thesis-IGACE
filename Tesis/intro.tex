\chapter{Introducción}

La adopción de tecnologías digitales ha agilizado el acceso al conocimiento, permitiendo que cada vez más personas puedan acceder a la educación \cite{UN2023ImpactDigitalTechnologies}. Sin embargo, este avance no está exento de desafíos, especialmente en el contexto de la educación en línea, donde surgen obstáculos adicionales que requieren soluciones innovadoras y adaptativas.

Un desafío presente en la educación contemporánea es mantener la atención de los estudiantes durante periodos prolongados. Aunque se plantea comúnmente que la capacidad de atención ha disminuido en las nuevas generaciones, diversos estudios contradicen dicha afirmación \cite{The_Role_of_Attention_Learning_Digital_Age}. Este contraste de opiniones refleja la polarización entre dos perspectivas extremas: los "doomsters", quienes tienen una visión pesimista sobre el impacto de la tecnología en la sociedad, y los "boosters", que adoptan una visión optimista frente al avance de esta \cite{Selwyn2014LookingF}.

Existe consenso en que la tecnología y su portabilidad han incrementado notablemente la exposición de los estudiantes a distracciones \cite{Zimmerman2011HandbookOS, Wang2022ComprehensivelySummarizeDistractions}. El término “multitasking” se refiere al intento de llevar a cabo múltiples tareas simultáneamente. Aunque las personas tienden a creer que son capaces de realizar varias actividades de manera concurrente, diversos estudios han demostrado que esta práctica conlleva una disminución en la productividad y en el proceso de aprendizaje \cite{Domoff2019AddictivePU}. Con la generalización de los dispositivos móviles y la adopción de clases en línea, se ha observado un aumento significativo en la tendencia al multitasking durante las clases \cite{Wang2022ComprehensivelySummarizeDistractions}.


Una forma de distracción reconocida en la literatura es la interferencia motivacional, concepto propuesto y explicado por Fries y Dietz \cite{Fries2007LearningMotivationalInterference}. El término indica que la motivación por los contenidos educativos disminuye ante la presencia de estímulos más atractivos, como los dispositivos móviles. Ante este panorama, surge la necesidad de explorar estrategias que mantengan la motivación y el enfoque de los estudiantes durante las clases.

Los videojuegos educativos surgen como estrategias válidas ante el escenario actual, como una potencial herramienta complementaria para la enseñanza, evaluación y entretenimiento de los estudiantes. Entre sus beneficios se destaca la motivación. De hecho, para disfrutar de una actividad, primero se debe permitir a la mente de un individuo percibir tal actividad como motivante \cite{Bisson1996FunInLEarningPedagogicalRole}.

Yu et al. \cite{Yu2020TheEffectsOfEducationGames} llevaron a cabo una revisión sistemática de la literatura sobre los efectos de los videojuegos educativos en el aprendizaje de los estudiantes y su motivación. En relación con este último aspecto, el estudio señala que la incorporación de videojuegos educativos como recurso complementario impacta positivamente en la motivación. Sin embargo, se menciona que la evidencia es insuficiente para concluir sobre la efectividad en los resultados del aprendizaje.

El equipo de Yu también menciona una ventaja asociada a los videojuegos educativos: pueden proporcionar servicios educacionales de alta calidad, flexibles, portátiles y de bajo costo, aumentando las interacciones entre materiales de aprendizaje, estudiantes y profesores \cite{Yu2020TheEffectsOfEducationGames}.

En el estudio mencionado, se afirma que el diseño del videojuego tiene una gran incidencia en el resultado final. Concluyen que las mecánicas, elementos visuales y narrativos tienen un efecto significativo, sugiriendo que los juegos educativos deberían implementar varios elementos de ludificación, como rankings, sistemas de recompensa, entre otros aspectos \cite{Yu2020TheEffectsOfEducationGames}.

En más del 50\% de los estudios citados anteriormente, se señala la falta de certeza con respecto a la percepción de los usuarios al jugar videojuegos educativos, concluyendo que se necesita una investigación más profunda. Por lo tanto, es pertinente emplear una prueba estandarizada que permita evaluar diversos diseños de videojuegos, abarcando diferentes poblaciones objetivo, y cuyos resultados sean medidos conforme a un estándar uniforme \cite{Yu2020TheEffectsOfEducationGames}. Con el fin de estandarizar los resultados, Petri et al. desarrolló el modelo MEEGA+ para medir la percepción de los usuarios respecto a un videojuego educativo \cite{meegaplus}.


Esta investigación surge a partir de los antecedentes planteados inicialmente, y en relación con los contenidos del campo de las ciencias de la computación. Se observa que, en muchas ocasiones, los estudiantes de esta área creen entender cómo funciona cierto código o algoritmo, sólo haciendo una revisión preliminar de las instrucciones, pero no hay una ejecución y lectura del paso a paso en paralelo. Estas prácticas llevan a que no se aprenda adecuadamente los contenidos, y se crea erradamente que se entienden a cabalidad los contenidos, pero posteriormente no logran extrapolarlos ni reproducirlos con otras variables, arrastrando conceptos erróneos sin percatarse \cite{IdentifyingStudentDifficultiesDataStructures}.

El trabajo busca tener un impacto positivo en el aprendizaje y la motivación de los estudiantes de computación con la aplicación del videojuego educativo IGASCE. Se enseñan contenidos de dos algoritmos vinculados a grafos. Estos son Búsqueda en Anchura (BFS) y Búsqueda en Profundidad (DFS). La idea del juego es represetar las instrucciones de algoritmos de grafos en una forma visual, interactiva y lúdica.

Se eligieron grafos y los algoritmos BFS y DFS porque poseen una representación visual directa y más fácil de diseñar en un videojuego. Visualizar los algoritmos permite una comprensión mayor a las explicaciones basadas en texto e ideas abstractas \cite{surti2023NeoRoute}.

Con esto, la investigación de tesis realiza una evaluación cuantitativa, mediante la medición de la motivación y percepción general del juego y el grado de aprendizaje adquirido, de los contenidos de algoritmos relacionados con grafos, en particular BFS y DFS. Además, existe una componente cualitativa basada en comentarios abiertos por parte de los participantes del estudio.

El público objetivo son estudiantes de primer año de ciencias de la computación, aunque también es aplicable a estudiantes de ingeniería con conocimientos de programación. El requisito principal es que no hayan estudiado grafos previamente. La evaluación se realiza mediante una prueba académica escrita, con un puntaje y nota; y el formulario MEEGA+, también con un puntaje asignable a cada criterio a evaluar.


\section{Trabajo Relacionado}

\subsection{Videojuegos educativos o serios}

Un videojuego representa una modalidad de aprendizaje activo, donde el proceso de enseñanza no sigue la estructura tradicional que consiste en tener a un profesor exponiendo frente a un estudiante. En este enfoque, el aprendizaje implica la ejecución de pasos y la participación activa del estudiante en actividades que contribuyen a la construcción del conocimiento. La revisión realizada por Hartikainen et al. \cite{active_learning_review} presenta argumentos a favor del aprendizaje activo, destacando mejores resultados, respaldo político y las demandas contemporáneas del entorno laboral, como habilidades de comunicación y la capacidad de aprendizaje autónomo.

Los videojuegos serios, también conocidos como ``Serious Gaming'', constituyen una forma de aprendizaje activo. Bell y Gibson \cite{evaluation_of_games_for_teaching_cs} sostienen que los juegos educativos son más efectivos que las clases tradicionales, lecturas, videos y tareas. Estos autores afirman que el uso de juegos conlleva a una mejora en la retención, conocimiento factual, habilidades basadas en el conocimiento, y autoeficacia. No obstante, recomiendan que los juegos deben complementarse con otras formas de enseñanza y actividades posteriores al juego que permitan a los estudiantes reflexionar sobre la relación entre los juegos y la materia.

\subsection{Análisis de otros autores}

Bell y Gibson \cite{evaluation_of_games_for_teaching_cs} identificaron y clasificaron 41 videojuegos de Ciencias de la Computación (CS). Uno de ellos, ``Map Coloring", aborda el tema de grafos, en particular el coloreo de grafos. Aunque se realizó una búsqueda del juego hasta la fecha (2022), no se encontró material al respecto.

Kiili y su equipo \cite{using_videogames_maths} analizaron el uso de videojuegos en enseñanza y evaluación en matemáticas a través de los títulos ``Semideus'' y "Wuzzit Trouble". A través de estos, llegaron a la conclusión de que es posible utilizar videojuegos para enseñar y evaluar al mismo tiempo. Además, caracterizaron estadísticamente las diferencias producidas por el uso de videojuegos entre resultados de un pre test y un post test.

% En otro trabajo de Kiili et. al \cite{Kiili_game_based_rational} realizaron una regresión lineal entre dos variables, puntaje obtenido en el videojuego y mejoras en una prueba académica relacionada con la materia enseñada. En este caso se enseñaba sobre fracciones y números racionales. En este caso, el tiempo dedicado por los grupos de estudio, fue un total de 2 horas y media. % TODO: revisar mejor la relación entre las variables.
% @article{Kiili_game_based_rational,
%   title = {Evaluating the effectiveness of a game-based rational number training - In-game metrics as learning indicators},
%   author = {Kristian Kiili and Korbinian Moeller and Manuel Ninaus},
%   journal = {Computers & Education},
%   volume = {120},
%   pages = {13-28},
%   year = {2018},
%   doi = {https://doi.org/10.1016/j.compedu.2018.01.012},
%   url = {https://www.sciencedirect.com/science/article/pii/S0360131518300125}
% }

En el trabajo de Zhao y Shute \cite{video_game_foster_computational_thinking}, se listan ejemplos de videojuegos diseñados para enseñar programación, como Wu's Castle \cite{wuscastle}, CodeCombat \cite{CodeCombat}, CodeSpell \cite{codespells}, y MiniColon \cite{minicolon}. Estos ejemplos emplean programación con texto. No obstante, también existen numerosos referentes que utilizan programación por bloques, como LightBot \cite{LightBot}, Scratch \cite{ scratch, maloney2010scratch}, y RoboBuilder \cite{RoboBuilder}, los cuales simplifican el proceso de aprender la sintaxis relacionada con los lenguajes de programación.

A pesar de esto, el impacto de estos videojuegos no ha sido evaluado rigurosamente en muchos casos. En las instancias documentadas, las muestras suelen ser reducidas, y las evaluaciones se centran principalmente en aspectos cualitativos, sin evaluar un aprendizaje determinado ni entregar números que permitan la comparación de estos juegos educativos con otros \cite{video_game_foster_computational_thinking, effectiveness_gbl}.

Petri y otros \cite{meegaplus} coinciden en la falta de sistematización en la evaluación de los videojuegos educativos. Se ha observado que algunos juegos serios ni siquiera se autodenominan o se consideran como tales \cite{evaluation_of_games_for_teaching_cs}. Además, muchos estudios carecen de resultados cuantitativos, lo que dificulta su comparación. 

Para abordar estas deficiencias, Petri et al. \cite{meegaplus} desarrollaron el modelo MEEGA+ (Modelo para la Evaluación de Juegos Educativos y Escala EGameFlow). Entre las deficiencias identificadas al momento de crear videojuegos según Petri et al. \cite{meegaplus} se encuentran la falta de: 
(1) Definición de un objetivo de evaluación; (2) Diseño de investigación; (3) Plan de medición; (4) Instrumentos de recopilación de datos; y (5) Métodos de análisis de datos. Una práctica común al analizar estas herramientas es utilizar comentarios informales por parte del estudiantado, en lugar de aplicar escalas psicométricas \cite{meegaplus}.

% Entre las deficiencias señaladas al momento de crear videojuegos según Petri et al. \cite{meegaplus} se encuentran la carencia de: 1) Definición de un objetivo de evaluación; 2) Diseño de investigación; 3) Plan de medición; 4) Instrumentos de recopilación de datos; y 5) Métodos de análisis de datos. Una práctica que era común al analizar estas herramientas es hacer uso de comentarios informales por parte del estudiantado, sin la aplicación de escalas psicométricas \cite{meegaplus}.


\section{Preguntas de Investigación}
\label{RQ1}
\emph{RQ1}: ¿Puede un grupo de estudiantes que no tenga conocimiento previo sobre grafos, identificar y construir un recorrido BFS y DFS solo habiendo sido expuesto a un videojuego educativo sobre grafos?

\label{RQ2}
\emph{RQ2}: ¿Cómo perciben los estudiantes de ciencias de la computación sin conocimientos de grafos un videojuego educativo sobre grafos?


\section{Hipótesis}

\label{Hip1}
\emph{H1}: Un videojuego que utilice conceptos relacionados a grafos puede enseñar sobre los algoritmos de BFS y DFS.

\label{Hip2}
\emph{H2}: Un videojuego que enseñe grafos será percibido de manera positiva por los estudiantes que todavía no aprenden sobre esos contenidos.


\section{Objetivo General}

El objetivo principal de este trabajo es medir el aprendizaje y la motivación de los estudiantes de computación al utilizar un videojuego educativo para instruir en algoritmos vinculados a grafos.

\section{Objetivos Específicos}

\begin{itemize}

\item Concebir una aplicación interactiva que visualice grafos y permita seguir los pasos relacionados con algoritmos que operan en dichos grafos. Esta aplicación debe tener elementos característicos de los videojuegos educativos.

\item Idear, desarrollar e implementar mecánicas de juego y una arquitectura de programación transferibles a otros videojuegos que instruyan en materias relacionadas con la programación.

\item Llevar a cabo una evaluación estandarizada para medir la percepción de los estudiantes respecto al videojuego creado.

\end{itemize}

