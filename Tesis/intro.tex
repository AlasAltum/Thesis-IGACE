\chapter{Introducción}

La presente investigación busca llevar las instrucciones llevadas a cabo por los algoritmos relacionados a grafos 
a una representación visual e interactiva para que los estudiantes de computación puedan aprender acompañándose con 
esta herramienta con el fin de determinar si el uso de herramientas interactivas y con feedback visual logra mejores 
resultados a nivel de aprendizaje y motivación en los estudiantes.

El trabajo consiste en presentar a estudiantes de ciencias de la computación un videojuego donde se muestren grafos, 
en que el jugador debe ejecutar las instrucciones de estos algoritmos, apoyándose en elementos visuales. 
El objetivo de esta investigación de tesis es determinar si se ven diferencias en los niveles de motivación percibidos por 
el estudiantado y en el entendimiento de los algoritmos, medido a través de una prueba donde se pregunta por algoritmos de grafos.
El público objetivo son estudiantes que estén en su primer año de ciencias de la computación (CS por sus siglas en inglés), 
aunque también es aplicable a estudiantes de otras carreras relacionadas a ingeniería que sepan de programación pero que
no hayan aprendido de grafos todavía.


\section{Motivación}

% Hablar que existen formas más eficientes de aprendizaje
% Revisar si el feedback inmediato ofrece ventajas sobre el feedback a largo plazo. 
El uso de tecnologías digitales ha permitido acelerar el acceso al conocimiento y su escabilidad, permitiendo que estudiantes que
habrían sido excluidos en otros contextos, ahora puedan acceder a la educación. Sin embargo, la educación en línea tiene sus propios
retos \cite{UN2023_ImpactDigitalTechnologies}.
En este contexto, se vuelve importante buscar metodologías que permitan acelerar el aprendizaje de los estudiantes con 
tecnologías adaptables, escalables, y que permitan un aprendizaje más eficiente.
Se postula popularmente que la capacidad de atención de forma prolongada ha disminuido en las nuevas generaciones.
Hay estudios que contradicen estas afirmaciones \cite{The_Role_of_Attention_Learning_Digital_Age}, indicando que las
habilidades cognitivas de los estudiantes han cambiado, pero no necesariamente empeorado. Existen términos como ``doomsters'' y ``boosters'' 
\cite{Selwyn2014LookingF}, para descibrir la polarización entre estas miradas con respecto a la tecnología.
En lo que sí existe consenso, es que la tecnología y su portabilidad ha incrementado la cantidad de distracciones a las que se exponen
los estudiantes \cite{Zimmerman2011HandbookOS, Wang2022ComprehensivelySummarizeDistractions}. 
A intentar realizar más de una tarea a la vez se le llama multitasking, y se ha demostrado que disminuye la productividad y el aprendizaje, 
aunque las personas creen que pueden hacer más de una cosa a la vez \cite{Domoff2019AddictivePU}.  Con la popularización de los 
smartphones y las clases en línea, el multitasking -desde hacer labores del hogar hasta sacar el celular y revisar alguna aplicación- durante clases se ha incrementado \cite{Wang2022ComprehensivelySummarizeDistractions}. 
Una forma de distracción reconocida en la literatura es la interferencia motivacional, propuesta y explicada por Fries y Dietz \cite{Fries2007LearningMotivationalInterference}. 
Esta indica que la motivación respecto a la clase disminuye debido a la presencia de tentaciones más atractivas para la atención, como los smartphones. 
En este contexto, donde los estudiantes se ven tentados a distraerse, es importante buscar formas de mantenerlos motivados y enfocados durante las clases. Es aquí donde
se proponen utilizar videojuegos educativos como una manera de mantener a los estudiantes motivados y enfocados en la tarea que se les pide realizar.
Estos se destacan porque tienen potencial como una herramienta complementaria para la enseñanza, evaluación y entretemiento para los estudiantes. 
Entre los beneficios que reportan los videojuegos educativos se destaca la motivación. En \cite{Bisson1996FunInLEarningPedagogicalRole} se indica
que para disfrutar una actividad, primero se debe permitir a la mente de un individuo percibir tal actividad como motivante. 
En \cite{Yu2020TheEffectsOfEducationGames}, Yu hace una revisión sistemática de la literatura sobre los efectos de los videojuegos educativos en el aprendizaje de los estudiantes y su motivación,
En este estudio, con respecto a la motivación, se señala que en un contexto de aprendizaje que utilice o se complemente con videojuegos educativos afecta positivamente 
la motivación e incluso los logros académicos, pero también indica que han habido estudios que contradicen esta afirmación, por lo que se requiere más investigación en el área.
Además, Yu también indica que los juegos logran resultados positivos en el aprendizaje, pero indicando que son utilizados como complementos a la enseñanza tradicional, y no como un reemplazo total,
además, se indica que el diseño del videojuego afecta totalmente el resultado final. Por ejemplo, los juegos de acción o realidad aumentada tenían mejores resultados que un
juego tradicional, y que las mecánicas, elementos visuales y narrativos afectan significativamente el resultado final.
Considerando estos antecedentes, de que un videojuego educativo puede enseñar, es escalable, repetible y flexible, es que se presenta una oportunidad para analizar la percepción
de estudiantes de computación con respecto a un videojuego educativo que enseñe algoritmos relacionados a grafos.
En la mayoría de los estudios citados previamente se indica que falta tener certeza respecto de la percepción de los usuarios al jugar videojuegos educativos y que 
se requiere indagar más. Por lo mismo, resulta conveniente utilizar la misma prueba estandarizada para probar distintos diseños de videojuegos,
con distintas poblaciones objetivo. Por esta razón, se utilizó el formulario MEEGA+ para medir la motivación de los estudiantes, y una prueba de conocimiento para 
medir el aprendizaje de los estudiantes.


\section{Objetivo General}

% Indicar aquí que queremos medir aprendizaje y motivación
El objetivo principal de este trabajo es medir el aprendizaje y la motivación de los estudiantes de computación al 
utilizar un videojuego educativo para enseñar algoritmos relacionados a grafos con las características presentadas en este trabajo.

\section{Objetivos Específicos}

% Determinar falencias, recomendaciones de diseño para un futuro videojuego
% Factores a considerar, como fecha en que se hace el estudio, la naturaleza de la muestra
% Su disponibilidad, etc
\begin{itemize}

\item Diseñar una aplicación interactiva que muestre grafos y permita seguir los pasos relacionados a
algoritmos que trabajen con grafos.

\item Aplicar una prueba estandarizada para medir la percepción de los estudiantes sobre el videojuego creado.

\item Proponer, diseñar e implementar una mecánica de juego y arquitectura de programación que permita ser transferida a 
otros videojuegos que enseñen materias relacionadas a la programación. 


\end{itemize}


